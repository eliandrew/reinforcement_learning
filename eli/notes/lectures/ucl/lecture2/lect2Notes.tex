\documentclass[12pt]{article}
\usepackage{fullpage,enumitem,amsmath,amssymb,graphicx}

\newcommand{\ub}{\boldsymbol{u}}
\newcommand{\vb}{\boldsymbol{v}}

\begin{document}

\begin{center}
{\Large Reinforcement Learning David Silver: Lecture 2 Notes}

\begin{tabular}{rl}
Name: & Eli Andrew
\end{tabular}
\end{center}

  \begin{itemize}
    \item Reward: 
    $G_t = R_{t+1} + \gamma R_{t+2} + \gamma^2 R_{t+3} = \sum_{k = 0}^{\infty}\gamma^k R_{t+k+1}$ \\
    This is a single sample of $G_t$ and therefore contains $R$ rather than $E[R]$.
    \item Discount factor is useful as a way of dealing with uncertainty in your model.
    Current rewards not just because they are in the present but because future rewards 
    are more uncertain due to the constraints on our model.  
    \item Discounting also avoids infinte rewards.
    \item Value function: $v(s)$ gives long-term value of state $s$.
    $v(s) = E[G_t | S_t = s]$
    \item Bellman Equation for MRPs: value function can be decomposed into two parts
    \begin{itemize}
      \item Immediate reward: $R_{t+1}$
      \item Discounted future reward: $\gamma v(S_{t+1})$
      \item Calculation: \\
      $v(s) = E[G_t | S_t = s] \\
      = E[R_{t+1} + \gamma R_{t+2} + \gamma^2 R_{t+3} + \dots | S_t = s] \\
      = E[R_{t+1} + \gamma (R_{t+ 2} + \gamma R_{t+3} + \dots ) | S_t = s] \\
      = E[R_{t+1} + \gamma G_{t+1} | S_t = s] \\
      = E[R_{t+1} + \gamma v(S_{t+1}) | S_t = s]$
    \end{itemize}
    \item Bellman Equation expressed using matrices: $v = R + \gamma Pv$
    where $v$ is a column vector with one entry per state.
    \item Solved directly the Bellman equation solution is $v = (I - \gamma P)^{-1}R$
    \item Iterative methods for large MDPs: (1) dynamic programming, (2) Monte-carlo
    simulation, (3) temporal-difference learning
    \item Markov Decision Process is a Markov Reward Process but with Actions. In other
    words the reward process is $(S, P, R, \gamma)$ and the decision process is $(S, A, P, R, \gamma)$
    \item Policy definition: a distribution over actions given states $\pi(a | s) = P[A_t = a | S = s]$
    \item Because of the Markov property, we do not need to consider $R$ in the policy because $s$ fully
    characterizes the evolution from this point onwards.
    \item The policy itself, with the states that it picks, defines a Markov process $(S, P^\pi)$, and 
    the state and rewards the policy draws defines a Markov reward process, where: \\
    $P^\pi_{(s, s')} = \sum_{a \in A}\pi(a | s)P^a_{s, s'}$ \\
    $R^\pi_{(s, s')} = \sum_{a \in A}\pi(a | s)R^a_{s}$
    \item State value function $v_\pi(s)$ of an MDP is the expected reward starting at state $s$
    and then following policy $\pi$: $v_\pi(s) = E_\pi[G_t | S_t = s]$
    \item Action value function $q_\pi(s,a)$ is the expected return starting from state $s$, taking action $a$,
    and then following policy $\pi$. $q_\pi(s,a) = E_\pi[G_t | S_t = s, A_t = a]$
    \item Decomposed state-value function: $v_\pi(s) = E_\pi[R_{t+1} + \gamma v_\pi(S_{t+1}) | S_t = s]$
    \item Decomposed action-value function: $q_\pi(s,a) = E_\pi[R_{t+1} + \gamma q_\pi(S_{t+1}, A_{t+1}) | S_t = s, A_t = a]$
    \item Bellman equation for $V^\pi$: $v_\pi(s) = \sum_{a \in A}\pi(a | s)q_\pi(s, a)$ where the policy is giving us the probability
    of taking the action $a$ given we're in state $s$ and the action-value function is giving us the value of the the action.
    \item Bellman equation for $Q^\pi$: $q_\pi(s, a) = R_s^a + \gamma \sum_{s' \in S}P_{ss'}^a v_\pi(s')$ where
    we are getting our immediate reward $R_s^a$ for the current state and then the discounted reward over all possible states $s' \in S$ we could end
    up in when taking action $a$ from state $s$.
  \end{itemize}


\end{document}
