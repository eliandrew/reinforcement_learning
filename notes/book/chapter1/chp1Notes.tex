\documentclass[12pt]{article}
\usepackage{fullpage,enumitem,amsmath,amssymb,graphicx}

\newcommand{\ub}{\boldsymbol{u}}
\newcommand{\vb}{\boldsymbol{v}}

\begin{document}

\begin{center}
{\Large Reinforcement Learning: Chapter 1 Notes}

\begin{tabular}{rl}
Name: & Eli Andrew
\end{tabular}
\end{center}

\section*{Section 1.1 Reinforcement Learning}

  \begin{itemize}
    \item The distinction between problems and solution methods is very
    important in reinforcement learning
    \item While reinforcement learning seems like a kind of unsupervised learning
    because it is not relying on examples of correct behavior, it is actually trying
    to maximize a reward signal instead of trying to find hidden structure.
    \item Methods like genetic algorithms never estimate value functions. They apply
    multiple static policies each interacting over an extended period of time with a
    separate isntance of the environment. The policies that obtain the most reward,
    and random variations of them, are carried over to the next generation of policies,
    and the process repeats.
    \item Evolutionary methods are effective in small policy search spaces and have advantages
    on problems in which the learning agent cannot sense the complete state of its enviornment.
    \item Temporal difference learning is named for changes based on difference $V(S_{t+1}) - V(S_t)$
    between estimates at two successive times.
    \item Evolutionary methods evaluate polcies by holding a given policy fixed and playing many games.
    The frequency of wins gives an unbiased estimate of the probability of winning with that policy, and
    can be used to direct the next policy selection.
    \item Evolutionary methods only change policy after serveral games and only the final outcome of each
    game is used: what happens \textit{during} the games is ignored. For example, if the player wins,
    then all of its behavior is given credit, independently of how specific moves might have been critical
    to the win. This can give credit to moves that never even occured.
    \item Value methods, in contrast to evolutionary methods, allow individual states to be evaluated. In the
    end, evolutionary and value function methods both search the space of policies, but learning a value function
    takes advantage of information available during the course of play.
  \end{itemize}


\end{document}
